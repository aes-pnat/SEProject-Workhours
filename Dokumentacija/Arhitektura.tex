\chapter{Arhitektura i dizajn sustava}
		
	Arhitektura se može podijeliti na tri podsustava: 
	\begin{itemize}
		\item Web poslužitelj 
		\item Web aplikacija 
		\item Baza podataka 
	\end{itemize}
	Web preglednik je program koji korisniku omogućuje pregled web stranica i multimedijalnih sadržaja vezanih uz njih. Svaki internetski preglednik je prevoditelj. Dakle, stranica je pisana u kodu koji preglednik nakon toga interpretira kako nešto svakome razumljivo. Korisnik putem web preglednika šalje zahtjev web poslužitelju. 
	
	Web poslužitelj osnova je rada web aplikacije. Njegova primarna zadaća je komunikacija klijenta s aplikacijom. Komunikacija se odvija preko HTTP (engl. Hyper Text Transfer Protocol) protokola, što je protokol u prijenosu informacija na webu. Poslužitelj je onaj koji pokreće web aplikaciju te joj prosljeđuje zahtjev. 
	
	Korisnik koristi web aplikaciju za obrađivanje željenih zahtjeva. Web aplikacija obrađuje zahtjev te ovisno o zahtjevu, pristupa bazi podataka nakon čega preko poslužitelja vraća korisniku odgovor u obliku HTML dokumenta vidljivog u web pregledniku. 
	
	Programski jezik kojeg smo odabrali za izradu naše web aplikacije je Java Spring Boot. Odabrano razvojno okruženje je Eclipse IDE. Arhitektura sustava temeljiti će se na MVC (Model-View-Controller) konceptu. 
	
	Karakteristika MVC koncepta je nezavisan razvoj pojedinih dijelova aplikacije što za posljedicu ima jednostavnije ispitivanje kao i jednostavnije razvijanje i dodavanje novih svojstava u sustav. \\
	
	MVC koncept sastoji se od: 
	\begin{itemize}
		\item Model - Središnja komponenta sustava. Predstavlja dinamičke strukture podataka, neovisne o korisničkom sučelju. Izravno upravlja podacima, logikom i pravilima aplikacije. Također prima ulazne podatke od Controllera 
		\item View - Bilo kakav prikaz podataka, poput grafa. Mogući su različiti prikazi iste informacije poput grafičkog ili tabličnog prikaza podatak. 
		\item Controller – Prima ulaze i prilagođava ih za prosljeđivanje Model-u ili View-u. Upravlja korisničkim zahtjevima i na temelju njih izvodi daljnju interakciju s ostalim elementima sustava. 
	\end{itemize}
		

		

				
		\section{Baza podataka}
			
			\textbf{\textit{dio 1. revizije}}\\
			
		\textit{Potrebno je opisati koju vrstu i implementaciju baze podataka ste odabrali, glavne komponente od kojih se sastoji i slično.}
		
			\subsection{Opis tablica}
			

				\textit{Svaku tablicu je potrebno opisati po zadanom predlošku. Lijevo se nalazi točno ime varijable u bazi podataka, u sredini se nalazi tip podataka, a desno se nalazi opis varijable. Svjetlozelenom bojom označite primarni ključ. Svjetlo plavom označite strani ključ}
				
				\begin{longtabu} to \textwidth {|X[6, l]|X[6, l]|X[20, l]|}
					
					\hline \multicolumn{3}{|c|}{\textbf{korisnik - ime tablice}}	 \\[3pt] \hline
					\endfirsthead
					
					\hline \multicolumn{3}{|c|}{\textbf{korisnik - ime tablice}}	 \\[3pt] \hline
					\endhead
					
					\hline 
					\endlastfoot
					
					\cellcolor{LightGreen}IDKorisnik & INT	&  	Lorem ipsum dolor sit amet, consectetur adipiscing elit, sed do eiusmod tempor incididunt ut labore et dolore magna aliqua. Ut enim ad minim veniam 	\\ \hline
					korisnickoIme	& VARCHAR &   	\\ \hline 
					email & VARCHAR &   \\ \hline 
					ime & VARCHAR	&  		\\ \hline 
					\cellcolor{LightBlue} primjer	& VARCHAR &   	\\ \hline 
					
					
				\end{longtabu}
			
			
			\subsection{Dijagram baze podataka}
				\textit{ U ovom potpoglavlju potrebno je umetnuti dijagram baze podataka. Primarni i strani ključevi moraju biti označeni, a tablice povezane. Bazu podataka je potrebno normalizirati. Podsjetite se kolegija "Baze podataka".}
			
			\eject
			
			
		\section{Dijagram razreda}
		
			\textit{Potrebno je priložiti dijagram razreda s pripadajućim opisom. Zbog preglednosti je moguće dijagram razlomiti na više njih, ali moraju biti grupirani prema sličnim razinama apstrakcije i srodnim funkcionalnostima.}\\
			
			\textbf{\textit{dio 1. revizije}}\\
			
			\textit{Prilikom prve predaje projekta, potrebno je priložiti potpuno razrađen dijagram razreda vezan uz \textbf{generičku funkcionalnost} sustava. Ostale funkcionalnosti trebaju biti idejno razrađene u dijagramu sa sljedećim komponentama: nazivi razreda, nazivi metoda i vrste pristupa metodama (npr. javni, zaštićeni), nazivi atributa razreda, veze i odnosi između razreda.}\\
			
			\textbf{\textit{dio 2. revizije}}\\			
			
			\textit{Prilikom druge predaje projekta dijagram razreda i opisi moraju odgovarati stvarnom stanju implementacije}
			
			
			
			\eject
		
		\section{Dijagram stanja}
			
			
			\textbf{\textit{dio 2. revizije}}\\
			
			\textit{Potrebno je priložiti dijagram stanja i opisati ga. Dovoljan je jedan dijagram stanja koji prikazuje \textbf{značajan dio funkcionalnosti} sustava. Na primjer, stanja korisničkog sučelja i tijek korištenja neke ključne funkcionalnosti jesu značajan dio sustava, a registracija i prijava nisu. }
			
			
			\eject 
		
		\section{Dijagram aktivnosti}
			
			\textbf{\textit{dio 2. revizije}}\\
			
			 \textit{Potrebno je priložiti dijagram aktivnosti s pripadajućim opisom. Dijagram aktivnosti treba prikazivati značajan dio sustava.}
			
			\eject
		\section{Dijagram komponenti}
		
			\textbf{\textit{dio 2. revizije}}\\
		
			 \textit{Potrebno je priložiti dijagram komponenti s pripadajućim opisom. Dijagram komponenti treba prikazivati strukturu cijele aplikacije.}