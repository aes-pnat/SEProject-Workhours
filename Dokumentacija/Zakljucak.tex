\chapter{Zaključak i budući rad}
		
% 		\textbf{\textit{dio 2. revizije}}\\
		
% 		 \textit{U ovom poglavlju potrebno je napisati osvrt na vrijeme izrade projektnog zadatka, koji su tehnički izazovi prepoznati, jesu li riješeni ili kako bi mogli biti riješeni, koja su znanja stečena pri izradi projekta, koja bi znanja bila posebno potrebna za brže i kvalitetnije ostvarenje projekta i koje bi bile perspektive za nastavak rada u projektnoj grupi.}
		
% 		 \textit{Potrebno je točno popisati funkcionalnosti koje nisu implementirane u ostvarenoj aplikaciji.}

        Zadatak našega tima bio je razvoj aplikacije koja bi omogućila poduzeću „Mi puno radimo“ praćenje realizacije zadataka te raspoloživosti djelatnika kako bi se olakšala organizacija djelatnosti i zaposlenika unutar firme. Kroz 15 tjedana razvoja uspješno smo ostvarili naš cilj.  

Prvi zadatak, okupljanje tima nije bio težak. Prethodna poznanstva uvelike su olakšala komunikaciju i pozitivno pridonijele funkcioniranju grupe. Zatim je uslijedila prva faza provedbe projekta. Zajednički smo pisali dokumentaciju što je olakšalo daljnji rad. Dokumentirani zahtjevi, izrađeni obrasci i dijagrami detaljno su prikazivali željene funkcionalnosti aplikacije. Svaki član tima, na temelju priloženog, znao je točno što treba implementirati te što je već bilo odrađeno. Konkretnu implementaciju zadataka možemo svrstati u drugu fazu razvoja. 

Početak je bio težak, no zajedničkim snagama uspjeli smo se uhodati u razvoj aplikacije. Grupa se podijelila na četiri \textit{backend} programera, dva \textit{frontend} te jednog zaduženog za bazu podataka. Daljnjim razvojem i obavljanjem nekih od poslova, potreba za reorganiziranjem tima bila je nužna, stoga spomenute pozicije nisu bile fiksne cijelo vrijeme razvoja. Završna situacija zahtijevala je više \textit{frontend} programera te nekoliko osoba za sređivanje dokumentacije. Ovakva reorganizacija tima pridonijela je da svaki člana nauči više. Ipak, reorganizacijom izgubili smo dosta vremena učeći nove alate i određene jezike, ali upravo se zato možemo još više ponositi postignutim ciljem. 

Za sve članove ovakav projekt bio je potpuno novo iskustvo.  Možemo istaknuti nova stečena znanja iz programiranja, no i usvojene vještine poput rada u timu, organizacije tima, ali i vremena. Prostor za poboljšanje uvijek postoji, no svakim sljedećim ovakvim iskušenjem uslijedit će i rast našeg iskustva, te će aplikacije biti bolje. 
		
		\eject 