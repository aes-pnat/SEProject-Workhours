\chapter{Specifikacija programske potpore}
		
	\section{Funkcionalni zahtjevi}
			
			
			\noindent \textbf{\\Dionici:}
			
			\begin{packed_enum}
				\item Vlasnik sustava (direktor poduzeća/naručitelj)
				\item Djelatnici poduzeća
				\begin{itemize}
					\item Voditelji grupa
					\item Ostali djelatnici
				\end{itemize}
				\item Razvojni tim
			\end{packed_enum}
			
			\noindent \textbf{Aktori i njihovi funkcionalni zahtjevi:}
			\begin{packed_enum}
				\item  \underbar{Vlasnik sustava (direktor poduzeća) može:}
				
				\begin{packed_enum}
					\item definirati uslužne djelatnosti koje će poduzeće raditi
					\item dodijeliti djelatnosti voditeljima grupa
					\item registrirati novog djelatnika (uključujući i voditelja grupa) i pritom mu dodijeliti ulogu
					\item vidjeti zauzetost i realizaciju za sve djelatnike
					\item vidjeti trenutno i prošlu poticiju svih djelatnika koji su izašli na intervencije na karti
				\end{packed_enum}
			
				\item  \underbar{Voditelji grupa mogu:}
				
				\begin{packed_enum}
					\item definirati zadatke i dodjeljivati ih djelatnicima
					\item zabilježiti procjenu radnih sati potrebnih za zadatak
					\item odrediti cijenu sata rada ovisno o djelatnosti ili zadatku
					\item pregledati podatke za sebe i svoju grupu
					\item upisati broj odrađenih sati za svaki dan
				\end{packed_enum}
				\eject
				\item  \underbar{Ostali djelatnici mogu:}
				\begin{packed_enum}
					\item vidjeti koji su mu zadaci dodijeljeni i u kojim se grupama nalazi
					\item pregledati vlastite podatke
					\item upisati broj odrađenih radnih sati za svaki dan
				\end{packed_enum}
			
			\item  \underbar{Neregistrirani korisnik može:}
			\begin{packed_enum}
				\item vidjeti popis i opis djelatnosti poduzeća
			\end{packed_enum}
			\end{packed_enum}
			
			\eject 

			\subsection{Obrasci uporabe}
				
				
				\subsubsection{Opis obrazaca uporabe}
					\textit{Funkcionalne zahtjeve razraditi u obliku obrazaca uporabe. Svaki obrazac je potrebno razraditi prema donjem predlošku. Ukoliko u nekom koraku može doći do odstupanja, potrebno je to odstupanje opisati i po mogućnosti ponuditi rješenje kojim bi se tijek obrasca vratio na osnovni tijek.}\\
					

					\noindent \underbar{\textbf{UC$<$broj obrasca$>$ -$<$ime obrasca$>$}}
					\begin{packed_item}
	
						\item \textbf{Glavni sudionik: }$<$sudionik$>$
						\item  \textbf{Cilj:} $<$cilj$>$
						\item  \textbf{Sudionici:} $<$sudionici$>$
						\item  \textbf{Preduvjet:} $<$preduvjet$>$
						\item  \textbf{Opis osnovnog tijeka:}
						
						\item[] \begin{packed_enum}
	
							\item $<$opis korak jedan$>$
							\item $<$opis korak dva$>$
							\item $<$opis korak tri$>$
							\item $<$opis korak četiri$>$
							\item $<$opis korak pet$>$
						\end{packed_enum}
						
						\item  \textbf{Opis mogućih odstupanja:}
						
						\item[] \begin{packed_item}
	
							\item[2.a] $<$opis mogućeg scenarija odstupanja u koraku 2$>$
							\item[] \begin{packed_enum}
								
								\item $<$opis rješenja mogućeg scenarija korak 1$>$
								\item $<$opis rješenja mogućeg scenarija korak 2$>$
								
							\end{packed_enum}
							\item[2.b] $<$opis mogućeg scenarija odstupanja u koraku 2$>$
							\item[3.a] $<$opis mogućeg scenarija odstupanja  u koraku 3$>$
							
						\end{packed_item}
					\end{packed_item}
				
					
				\subsubsection{Dijagrami obrazaca uporabe}
					
					\textit{Prikazati odnos aktora i obrazaca uporabe odgovarajućim UML dijagramom. Nije nužno nacrtati sve na jednom dijagramu. Modelirati po razinama apstrakcije i skupovima srodnih funkcionalnosti.}
				\eject		
				
			\subsection{Sekvencijski dijagrami}
				
				\textbf{\textit{dio 1. revizije}}\\
				
				\textit{Nacrtati sekvencijske dijagrame koji modeliraju najvažnije dijelove sustava (max. 4 dijagrama). Ukoliko postoji nedoumica oko odabira, razjasniti s asistentom. Uz svaki dijagram napisati detaljni opis dijagrama.}
				\eject
	
		\section{Ostali zahtjevi}
		 	
		 	\noindent Sustav treba:
			 \begin{itemize}
			 	\item omogućiti korištenje hrvatskih dijakritičkih znakova pri unosu i prikazu tekstualnog sadržaja
			 	\item podržati višekorisnički rad u realnom vremenu
			 	\item dati odgovor na traženi upit unutar nekoliko sekundi kada se dohvaćaju podaci iz baze podataka
			 	\item imati intuitivno i jednostavno za korištenje korisničko sučelje
			 	\item biti implementiran kao web aplikacija koristeći objektno-orijentirane jezike
			 	\item neispravno korištenje korisničkog sučelja ne smije narušiti rad sustava
			 	\item osigurati sigurnu, brzu i otpornu na vanjske greške vezu s bazom podataka
			 \end{itemize}
			 
			 
			 
	